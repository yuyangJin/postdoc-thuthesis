% !TeX root = ../thuthesis-example.tex

\begin{resume}

  \section*{个人简历}

  1995 年 12 月 19 日出生于浙江省江山市。

  2013 年 9 月考入北京理工大学计算机科学与技术系物联网工程专业,2017 年 6 月本科毕业并获得工学学士学位。

  2017 年 9 月免试进入清华大学计算机科学与技术系高性能计算研究所,2022 年 6 月博士毕业并获得工学博士学位。

  2022 年 7 月留校继续在清华大学计算机科学与技术系高性能计算研究所从事博士后研究工作至今。

  \section*{博士期间完成的相关学术成果}

  \subsection*{学术论文}

  \begin{achievements}
    \item \textbf{Jin Y}, Wang H, Zhong R, et al. PerFlow: A domain specific framework for automatic performance analysis of parallel applications[C]//Proceedings of the 27th ACM SIGPLAN Symposium on Principles and Practice of Parallel Programming. 2022: 177-191.
    \item \textbf{Jin Y}, Wang H, Yu T, et al. ScalAna: Automating scaling loss detection with graph analysis[C]//SC20: International Conference for High Performance Computing, Networking, Storage and Analysis. IEEE, 2020: 1-14.
    \item Zhai J, Zheng L, Zhang F, et al. Detecting performance variance for parallel applications without source code[J]. IEEE Transactions on Parallel and Distributed Systems, 2022, 33(12): 4239-4255.
    \item Zheng L, Zhai J, Tang X, et al. Vapro: Performance variance detection and diagnosis for production-run parallel applications[C]//Proceedings of the 27th ACM SIGPLAN Symposium on Principles and Practice of Parallel Programming. 2022: 150-162.
    \item 傅天豪, 田鸿运, \textbf{金煜阳}, 杨章, 翟季冬, 武林平, 徐小文. 一种面向构件化并行应用程序的性能骨架分析方法[J]. 计算机科学, 2021, 48(6): 1-9.
  \end{achievements}


  \subsection*{专利}

  \begin{achievements}
    \item 翟季冬, \textbf{金煜阳}, 陈文光, 郑纬民. 并行程序可扩展性瓶颈检测方法和计算装置: 中国, 202080035153.3(专利申请号)[P]. 2021-11-30.
    \item 翟季冬, \textbf{金煜阳}, 钟闰鑫, 王豪杰. 性能分析编程框架、方法和装置: 中国, 202210105952.4(专利申请号)[P]. 2022-01-28.
  \end{achievements}

  \subsection*{获奖情况}

  \begin{achievements}
    \item 清华大学优秀博士学位论文. 清华大学. 2022.
    \item 清华大学优秀博士毕业生. 清华大学. 2022.
  \end{achievements}

  \section*{博士后期间完成的相关学术成果}

  \subsection*{学术论文}

  \begin{achievements}
    \item \textbf{Jin Y}, Shui X, Zhai M, et al. TraceFlow: Efficient Trace Analysis for Large-Scale Parallel Applications via Interaction Pattern-Aware Trace Distribution[C]. (已被SC25录用)
    \item \textbf{Jin Y}, Wang H, Tang X, et al. Leveraging Graph Analysis to Pinpoint Root Causes of Scalability Issues for Parallel Applications[J]. IEEE Transactions on Parallel and Distributed Systems, 2024.
    \item \textbf{Jin Y}, Zhong R, Long S, et al. Efficient Inference for Pruned CNN Models on Mobile Devices With Holistic Sparsity Alignment[J]. IEEE Transactions on Parallel and Distributed Systems, 2024.
    \item \textbf{Jin Y}, Wang H, Zhong R, et al. Graph-Centric Performance Analysis for Large-Scale Parallel Applications[J]. IEEE Transactions on Parallel and Distributed Systems, 2024.
    \item \textbf{Jin Y}, Ma Z, Zhai J. Efficient Asynchronous Performance Prediction for Heterogeneous Systems[J]. Chinese Journal of Computational Physics, 2024, 41(1): 40.
    \item Huang K, Zhu S, Zhai M, et al. mTuner: Accelerating Parameter-Efficient Fine-Tuning with Elastic Tensor[C]. (已被USENIX ATC 25录用)
    \item Zhong R, \textbf{Jin Y}, Zhang C, et al. FlashTensor: Optimizing Tensor Programs by Leveraging Fine-grained Tensor Property[C]//Proceedings of the 30th ACM SIGPLAN Annual Symposium on Principles and Practice of Parallel Programming. 2025: 183-196.
    \item Zhou Y, Zong Y, \textbf{Jin Y}, et al. An Efficient 2D Fusion Method for High-Performance Two-Stage Eigensolvers on Modern Heterogeneous Architectures[J]. 2025.
    \item Lei K, \textbf{Jin Y}, Zhai M, et al. {PUZZLE}: Efficiently Aligning Large Language Models through {Light-Weight} Context Switch[C]//2024 USENIX Annual Technical Conference (USENIX ATC 24). 2024: 127-140.
    \item Huang K, Zhai J, Zheng L, et al. WiseGraph: Optimizing GNN with joint workload partition of graph and operations[C]//Proceedings of the Nineteenth European Conference on Computer Systems. 2024: 1-17.
  \end{achievements}


  \subsection*{专著}

  \begin{achievements}
    \item Zhai J, \textbf{Jin Y}, Chen W, et al. Performance Analysis of Parallel Applications for HPC[M]. Springer, 2023.
    \item \textbf{金煜阳}. 大规模并行程序性能分析与优化关键技术研究. 清华大学出版社, 2024.
  \end{achievements}

  \subsection*{获奖情况}

  \begin{achievements}
    \item 清华大学``水木学者". 清华大学. 2022.
    \item 中国计算机学会体系结构优秀博士学位论文激励计划. 中国计算机学会. 2023.
    \item 博士后创新人才支持计划. 中国博士后科学基金会. 2023.
  \end{achievements}

\end{resume}



% 本科生格式:

% \begin{resume}
%   \section*{学术论文}
%
%   \begin{achievements}
%     \item ZHOU R, HU C, OU T, et al. Intelligent GRU-RIC Position-Loop
%       Feedforward Compensation Control Method with Application to an
%       Ultraprecision Motion Stage[J], IEEE Transactions on Industrial
%       Informatics, 2024, 20(4): 5609-5621.
%
%     \item 杨轶, 张宁欣, 任天令, 等. 硅基铁电微声学器件中薄膜残余应力的研究[J].
%       中国机械工程, 2005, 16(14):1289-1291.
%
%     \item YANG Y, REN T L, ZHU Y P, et al. PMUTs for handwriting recognition.
%       In press[J]. (已被Integrated Ferroelectrics录用)
%
%   \end{achievements}
%
%
%   \section*{专利}
%
%   \begin{achievements}
%     \item 胡楚雄, 付宏, 朱煜, 等. 一种磁悬浮平面电机: ZL202011322520.6[P]. 2022-04-01.
%
%     \item REN T L, YANG Y, ZHU Y P, et al. Piezoelectric micro acoustic sensor
%       based on ferroelectric materials: No.11/215, 102[P]. (美国发明专利申请号.)
%
%   \end{achievements}
% \end{resume}
