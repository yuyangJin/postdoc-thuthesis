\chapter{引言}

\section{研究背景与意义}
随着人工智能技术在算法创新与算力突破的双重驱动下飞速发展,以大语言模型(简称“大模型”)为代表的核心技术成果,已在语音识别、自然语言处理、图像识别等通用领域实现深度应用。
这些基座大模型凭借强大的跨模态理解能力和知识泛化能力,构建了智能化应用的底层技术框架。
然而,当面向法律、医疗、金融等专业领域时,通用大模型存在显著的能力缺口——法律文书需要精准的法条匹配,医疗诊断要求严格的临床逻辑,金融分析涉及复杂的合规约束,这些垂直领域的特殊需求倒逼技术体系向专业化、精细化方向演进。

人工智能后训练技术作为衔接通用模型与领域应用的关键桥梁,通过对基座大模型进行针对性优化,能够有效提升其在特定领域的任务表现。
例如在医疗领域,通过后训练可使大模型掌握电子病历的结构化解析能力;在金融场景,能赋予模型合规风险的实时研判能力。
此技术路径不仅避免了从头训练领域模型的高昂成本,更突破了通用模型的能力边界,成为扩大人工智能应用领域的核心技术手段。

\subsection{后训练的常用方法}
具体而言,后训练涵盖微调、对齐、适配器训练等多种技术形态,可实现模型参数的领域适配、输出行为的合规校准以及计算效率的定向优化。

\textbf{全量微调}:
全量微调是后训练技术中最基础且直观的方法。
该方法基于这样一个逻辑:预训练模型在大规模通用数据上学习到了丰富的基础特征和语义表示,通过对其所有参数进行调整,可以让模型在特定任务和领域数据上,重新学习并适应新的分布和规则。
在数据充足的场景下,全量微调的优势得以充分体现。
例如在图像识别领域,如果拥有海量的特定类别图像标注数据,全量微调能够让预训练的视觉模型(如 ResNet、ViT)对这些特定类别进行精准识别,甚至可以捕捉到细微的特征差异。
全量微调允许模型对所有参数进行深度调整,从而充分挖掘模型的潜力,在很多情况下能够实现最优的任务性能。
但随着模型规模不断扩大,特别是在面对千亿级参数模型时,全量微调面临巨大挑战。
每一个参数的更新都需要计算梯度并进行存储,这使得计算资源需求呈指数级增长。
不仅需要强大的智算集群来支持并行计算,还需要海量的内存来存储参数和中间计算结果,高昂的计算成本和漫长的训练时间成为其大规模应用的主要阻碍。


\textbf{高效参数微调}:
鉴于全量微调的高成本问题,高效参数微调(Parameter-Efficient Fine-Tuning, PEFT)技术成为学术界和工业界研究的重点方向。
这类技术的核心在于,通过仅对模型中少量关键参数进行调整,在尽量保持模型性能的前提下,大幅降低计算开销,实现性能与效率的平衡。
(1) 适配器训练是高效微调的典型代表。
它在预训练模型的架构中插入轻量级的适配器模块,这些模块通常由少量的神经网络层组成,专门用于学习特定任务的特征。
在训练过程中,适配器模块的参数被更新,而预训练模型的主干参数则保持冻结状态。
这样一来,既能够让模型学习到新任务的知识,又避免了对庞大的主干参数进行调整,显著减少了计算量和内存需求。
(2) 提示学习则从另一个角度进行创新,它无需修改模型参数,而是通过设计和优化输入提示,来引导模型生成符合需求的输出。
在自然语言处理任务中,可以在输入文本前添加特定的提示语句,让模型根据这些提示进行针对性的回答或生成。
这些提示可以是离散的文本片段,也可以是连续的可学习向量,通过优化提示的形式和内容,能够有效提升模型在特定任务上的表现 。
(3) 低秩适应(LoRA)技术则利用矩阵的低秩分解特性,对参数更新进行近似 。
在微调过程中,LoRA 为模型的权重矩阵添加低秩的参数矩阵,通过更新这些低秩矩阵来间接调整模型的行为,相比直接更新全部参数,其计算复杂度和内存占用大幅降低,同时在很多任务上能够达到与全量微调相近的效果。

\textbf{模型对齐}:
在人工智能技术的实际应用中,确保模型输出符合人类价值观与伦理规范至关重要,对齐技术正是解决这一问题的关键手段。
(1) 基于人类反馈的强化学习(RLHF)目前已成为主流的对齐方法。
它的基本流程是首先收集人类对模型输出的评估数据,根据这些数据构建一个奖励模型,该模型用于评估模型输出与人类期望的符合程度。
然后,利用强化学习算法,以奖励模型给出的分数作为反馈信号,对模型进行优化,促使模型生成更符合人类价值观的输出。
这种方法在大语言模型的对齐中取得了显著成效,使得模型能够避免生成有害、偏见或错误的内容。
(2) 监督微调(SFT)则是一种更为直接的方法,它通过收集大量标注好的合规数据,按照标准的监督学习方式对模型进行训练。
在训练过程中,模型学习如何根据输入生成符合规范和要求的输出,从而提升输出的可靠性和安全性。
通过 SFT,可以让模型在特定领域中遵循既定的规则和标准,例如在医疗领域输出符合诊疗规范的建议,在金融领域生成合规的分析报告。

\subsection{后训练场景的特征分析}

后训练与常规训练在技术特性和应用逻辑上存在显著差异,尤其在领域数据处理、参数调整策略及算法流程设计方面展现出独特特征。

(1) \textit{领域训练数据序列增长}。
在常规训练阶段,模型通常基于大规模通用数据(如处理后的文本、图像)进行训练,数据序列长度相对固定且分布均匀。
与之不同,后训练聚焦于垂直领域应用,数据序列呈现显著的长程依赖特性与变长特征。
以法律文书处理为例,合同文本可能包含数万字条款,医疗病历涉及患者多年诊疗记录的长文本描述;
金融领域的交易日志、审计报告也存在大量时序数据与复杂嵌套结构。
长序列数据带来双重挑战:一方面,传统注意力机制在处理超长序列时存在计算复杂度呈平方增长的问题;
另一方面,长程依赖关系的建模要求模型具备更强的上下文理解能力。
% 因此,后训练需针对性优化模型架构(如采用稀疏注意力机制)和训练策略(如分块计算),以适配领域数据的独特形态。

(2) \textit{模型参数更新与冻结}。
常规训练旨在使模型学习通用知识与基础特征,通常对所有参数进行全局更新,以最大化模型的表征能力。而后训练阶段,考虑到预训练模型已积累丰富的通用语义知识,为避免灾难性遗忘并提升效率,常采用参数冻结策略。具体而言,仅对模型中少量关键组件(如适配器模块、提示向量、特定层参数)进行调整,而保留预训练参数不变。
例如,在适配器训练中,预训练的Transformer主干参数保持冻结,仅更新插入的轻量级适配器网络;LoRA技术通过低秩矩阵分解近似参数更新,仅优化新增的低秩矩阵参数。
这种参数选择性更新策略带来三大优势:一是大幅降低计算开销与内存需求;
二是避免过度拟合领域数据,确保模型泛化能力;
三是实现快速适配,使模型在少量领域数据下即可完成优化。

(3) \textit{算法流程引入不同阶段多样负载}。 
常规训练以最小化任务损失为单一目标,训练流程相对直接。
后训练则通过生成、推理、训练等多阶段协同实现精细化模型优化。
生成阶段强调快速产出候选数据,采用轻量化计算提升效率;
推理阶段专注于模型性能评估与反馈信号生成,对计算精度要求更高;
训练阶段需平衡多目标优化,面临复杂的参数更新与分布式计算挑战。
各阶段计算负载、资源需求与技术侧重不同,需动态适配资源以实现模型的领域深度优化。 



\section{后训练场景面临的关键性能问题}

尽管后训练技术在理论研究与工程实践中取得显著进展,但其技术体系仍存在多方面问题,这些问题相互交织,形成制约领域模型落地的瓶颈。
根据上述后训练技术的三大特点,本文对应总结出后训练场景中对应在计算、内存和通信方面的三大重要性能问题:

(1) \textbf{领域训练数据序列增长导致算子计算效率低}:
相比常规训练中相对固定、较短的数据序列,长序列数据使得传统算子难以高效执行。
例如,在处理长文本时,注意力机制的计算复杂度会随序列长度呈平方增长,导致计算资源消耗急剧上升,计算时间大幅增加。
而为了减少计算量而诞生的注意力变体,则因为稀疏访存等问题,同样无法达到较高的计算效率。
传统的算子优化策略无法充分利用长序列数据的特性,难以对计算过程进行有效加速,使得算子计算效率成为后训练性能提升的一大阻碍。

(2) \textbf{模型参数更新策略导致内存利用率不足}:
后训练采用独特的模型参数更新与冻结策略,为降低计算成本,通常仅对少量关键参数进行更新,而冻结大部分预训练参数。
然而,这种策略导致训练系统在内存管理方面的问题。
固定权重的存储需求相对稳定,但动态激活值的内存占用会随着数据输入和计算过程发生较大变化。
由于内存分配策略难以灵活适应这种动态变化,要么会预留过多内存空间导致资源浪费,要么在遇到高维激活值时出现内存不足的情况。
例如,在处理长序列数据时,激活值张量的峰值内存可能远超平均值,传统固定的内存分配方式容易引发内存溢出,迫使训练中断,整体内存利用率难以达到理想水平。

(3) \textbf{算法流程多阶段负载导致阶段间切换通信开销大}:
后训练的算法流程引入生成、推理、训练等多个阶段,每个阶段的负载特性差异显著。
生成阶段注重高并发数据产出,推理阶段强调高精度评估计算,训练阶段则侧重于复杂的参数更新。
在不同阶段切换过程中,需要进行大量的参数和数据交互、同步。
例如,从生成阶段到推理阶段,需要将生成的大量数据传输至推理模块进行评估;
从推理阶段到训练阶段,又要将评估结果和相关数据传递给训练模块以指导参数更新。
频繁的阶段切换使得数据传输量剧增,在分布式训练环境下,数据在不同计算节点之间的传输会占用大量网络带宽,导致通信开销大幅增加,严重影响后训练的整体效率。


\section{本文的主要研究内容与贡献}

\subsection{本文的主要贡献}
针对上述挑战,本文围绕后训练系统的性能优化展开系统性研究,针对计算、内存、通信三方面开展了优化工作。
本文的创新和主要贡献包括:

1. 在算子计算效率优化方面,设计并实现了一个基于细粒度张量属性的算子编译系统 FlashTensor,通过深度解析张量的归约维度、数据类型等底层特征,扩大了优化空间,实现算子拆分与融合策略的动态搜索。
该方法突破了传统编译优化的粗粒度处理局限,在降低访存开销的同时,提升算子计算并行度,显著提升算子执行效率。

2. 在内存资源利用优化方面,提出了一个基于弹性张量的大模型微调内存管理系统 mTuner,通过动态感知训练过程中的张量生命周期,实时调整冻结权重与动态激活值的内存分配策略,显著提升现存利用率,从而减少部分通信和计算量,最终实现端到端高效参数微调的性能提升。

3. 在分布式通信优化方面,提出了一个基于轻量级上下文切换的后训练通信方 PUZZLE,针对生成、推理、训练等不同阶段的计算特性,设计差异化的并行策略与数据传输策略,减少了不同阶段之间切换时的通信开销,提升了大模型对齐的整体性能和效率。

\subsection{本文的组织及各章内容简介}
全文共分为六个章节,每一章的具体组织如下:
第一章概述了面向后训练的系统优化的研究背景与意义,然后介绍了目前后训练场景的特征以及面临的主要性能问题,最后介绍了本文的主要贡献。
第二章系统梳理后训练技术的研究现状,重点分析重点分析了国内外研究单位在人工智能算子优化、内存管理、分布式通信方面的研究现状。
第三章详细阐述 FlashTensor 方法的技术原理,包括张量属性识别器、细粒度张量属性感知优化及相关的实验评估。
第四章深入讨论 mTuner 的内存优化机制,涉及弹性张量的定义、生命周期管理、内存自适应执行计划及相关的实验评估。
第五章全面介绍 PUZZLE 的通信优化方案,涵盖基于亲和性的阶段内切换、基于相似性的阶段间切换及相关的实验评估。
最后,第六章总结了研究成果。