% !TeX root = ../thuthesis-example.tex

\chapter{总结}

本研究围绕大语言模型后训练过程中在算子计算效率低、内存利用不足和通信开销大关键问题,提出了三个创新系统,有效提升了模型后训练效率。

针对大模型后训练长序列领域数据导致的中间张量内存开销大、算子计算效率低问题,本文提出了细粒度张量属性感知的编译优化系统FlashTensor。
该系统先总结出归约依赖、广播、大小和值四个张量关键属性,设计张量属性识别器进行准确捕获;
再基于属性开展代数等价图变换、非凸内核映射等优化,并通过轻量级搜索获取最优方案。
实验显示,相比八种先进方法,FlashTensor在H100 GPU上的端到端性能和核心模块性能平均加速比达1.50倍和3.24倍,在A100 GPU上分别为1.86倍和3.70倍。

针对高效参数微调时冻结参数导致内存利用率不足的问题,本文构建了基于弹性张量的内存管理系统mTuner。
通过分析后训练PEFT微调内存使用情况,本研究发现权重和激活内存可动态调整,进而提出弹性张量概念,让张量大小能灵活变化以提升内存利用率。
基于此,mTuner可感知内存状态并制定自适应执行计划。
实验表明,相较于先进训练系统,mTuner在各类大语言模型上显著提升性能,吞吐量最高提升1.51倍(平均1.28倍),还能在3小时内完成Llama-2-13B模型微调,增强其数学能力。

针对大模型对齐时异构上下文高效切换难题,本文提出了基于上下文高效切换的通信优化系统PUZZLE。
通过抽象上下文概念,从阶段内和阶段间两个维度优化:阶段内提出基于亲和性的时间共享切换方案与混合方案,利用上下文亲和性实现计算重叠,减少切换空闲时间与开销;阶段间制定基于相似性的切换策略,降低通信和参数重组成本。
实验结果表明,与先进的RLHF训练系统DeepSpeed-Chat相比,PUZZLE在不同模型规模和GPU配置下性能显著提升,最高加速比达2.12倍,同时有效降低上下文切换开销,扩展性良好。 



% 模板支持 BibTeX 和 BibLaTeX 两种方式处理参考文献。
% 下文主要介绍 BibTeX 配合 \pkg{natbib} 宏包的主要使用方法。


% \section{顺序编码制}

% 在顺序编码制下,默认的 \cs{cite} 命令同 \cs{citep} 一样,序号置于方括号中,
% 引文页码会放在括号外。
% 统一处引用的连续序号会自动用短横线连接。

% \thusetup{
%   cite-style = super,
% }
% \noindent
% \begin{tabular}{l@{\quad$\Rightarrow$\quad}l}
%   \verb|\cite{zhangkun1994}|               & \cite{zhangkun1994}               \\
%   \verb|\citet{zhangkun1994}|              & \citet{zhangkun1994}              \\
%   \verb|\citep{zhangkun1994}|              & \citep{zhangkun1994}              \\
%   \verb|\cite[42]{zhangkun1994}|           & \cite[42]{zhangkun1994}           \\
%   \verb|\cite{zhangkun1994,zhukezhen1973}| & \cite{zhangkun1994,zhukezhen1973} \\
% \end{tabular}


% 也可以取消上标格式,将数字序号作为文字的一部分。
% 建议全文统一使用相同的格式。

% \thusetup{
%   cite-style = inline,
% }
% \noindent
% \begin{tabular}{l@{\quad$\Rightarrow$\quad}l}
%   \verb|\cite{zhangkun1994}|               & \cite{zhangkun1994}               \\
%   \verb|\citet{zhangkun1994}|              & \citet{zhangkun1994}              \\
%   \verb|\citep{zhangkun1994}|              & \citep{zhangkun1994}              \\
%   \verb|\cite[42]{zhangkun1994}|           & \cite[42]{zhangkun1994}           \\
%   \verb|\cite{zhangkun1994,zhukezhen1973}| & \cite{zhangkun1994,zhukezhen1973} \\
% \end{tabular}



% \section{著者-出版年制}

% 著者-出版年制下的 \cs{cite} 跟 \cs{citet} 一样。

% \thusetup{
%   cite-style = author-year,
% }
% \noindent
% \begin{tabular}{@{}l@{$\Rightarrow$}l@{}}
%   \verb|\cite{zhangkun1994}|                & \cite{zhangkun1994}                \\
%   \verb|\citet{zhangkun1994}|               & \citet{zhangkun1994}               \\
%   \verb|\citep{zhangkun1994}|               & \citep{zhangkun1994}               \\
%   \verb|\cite[42]{zhangkun1994}|            & \cite[42]{zhangkun1994}            \\
%   \verb|\citep{zhangkun1994,zhukezhen1973}| & \citep{zhangkun1994,zhukezhen1973} \\
% \end{tabular}

% \vskip 2ex
% \thusetup{
%   cite-style = super,
% }
% 注意,引文参考文献的每条都要在正文中标注
% \cite{zhangkun1994,zhukezhen1973,dupont1974bone,zhengkaiqing1987,%
%   jiangxizhou1980,jianduju1994,merkt1995rotational,mellinger1996laser,%
%   bixon1996dynamics,mahui1995,carlson1981two,taylor1983scanning,%
%   taylor1981study,shimizu1983laser,atkinson1982experimental,%
%   kusch1975perturbations,guangxi1993,huosini1989guwu,wangfuzhi1865songlun,%
%   zhaoyaodong1998xinshidai,biaozhunhua2002tushu,chubanzhuanye2004,%
%   who1970factors,peebles2001probability,baishunong1998zhiwu,%
%   weinstein1974pathogenic,hanjiren1985lun,dizhi1936dizhi,%
%   tushuguan1957tushuguanxue,aaas1883science,fugang2000fengsha,%
%   xiaoyu2001chubanye,oclc2000about,scitor2000project%
% }。
